\documentclass{beamer}
% Choose your desired theme
\usetheme{Boadilla}
\usepackage[style=numeric]{biblatex}

\addbibresource{./cabi.bib}
\title{Pedaling Towards Progress}
\subtitle{Analyzing Washington, DC's Bikesharing system using Open-source tools}
\author{Maxwell Lindsay}
\institute{Van Oord}
\date{\today}

\begin{document}

\begin{frame}
    \titlepage
\end{frame}
\section{Introduction}
\begin{frame}
    % consider splitting into a few slides with more pictures
    \frametitle{Introduction to CaBi}
    \begin{columns}
        \column{0.5\textwidth}
        \begin{figure}
            \includegraphics[]{800px-VA_07_2012_Capital_Bikeshare_4152.JPG}
            % \caption{By Mariordo (Mario Roberto Durán Ortiz) - Own work, CC BY-SA 3.0, https://commons.wikimedia.org/w/index.php?curid=20462784}
        \end{figure}
        \column{0.5\textwidth}
        \begin{itemize}
            \item Memberships are available for \$95 a year for unlimited trips under 45 minutes \cite{cabisite}. Casual users can pay \$1 to unlock the bike, then \$0.05 per minute.
            \item Primarily intended for short trips
            \item Docked bikeshare
            \item 700+ docks, ~35 million trips (so far)
        \end{itemize}

    \end{columns}
    \small{* recently, non-docked E-bikes that have been introduced}
\end{frame}

\begin{frame}
    \frametitle{The data}

    Capital Bikeshare publishes the following data about each trip on a monthly basis:

    \begin{itemize}
        \item start time
        \item end time
        \item start station index
        \item end station index
        \item Membership status of the rider (Member or Non-member?)
    \end{itemize}

These data are available as csv files per month or year from Capital Bikeshare at https://ride.capitalbikeshare.com/system-data
\end{frame}

\begin{frame}
    \frametitle{My goal}

    
\end{frame}

\section{Methodology}
\begin{frame}
    \frametitle{Removing invalid trips}
    % sankey diagram?
    \section{What makes a trip invalid?}
    
    Longer than a 4 hours

    This is an arbitrary threshold. However, the CaBi system is primarily intended for short trips, and the pricing reflects this. The bikes can be used for leisure and tourism, but they are priced to encourage users to change bikes regularly. Also, for the purposes of this project, long trips are less likely to have 


    Starting and ending at the same station 

    For obvious reasons, cannot easily be routed
a
    Stations with an invalid start or end station

    


    
\end{frame}

\begin{frame}
    \frametitle{Conclusion}
    This is the conclusion slide.
\end{frame}

\end{document}
